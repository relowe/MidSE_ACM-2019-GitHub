\documentclass[handout]{beamer}
\mode<presentation>
{
  \usetheme{Warsaw}
  \definecolor{mcgarnet}{rgb}{0.38, 0, 0.08}
  \definecolor{mcgray}{rgb}{0.6, 0.6, 0.6}
  \setbeamercolor{structure}{fg=mcgarnet,bg=mcgray}
  %\setbeamercovered{transparent}
}


\usepackage[english]{babel}
\usepackage[latin1]{inputenc}
\usepackage{times}
\usepackage[T1]{fontenc}
\usepackage{tikz}
\usepackage{graphicx}
\usepackage{fancyvrb}
\usepackage{adjustbox}

\newcommand{\imagesource}[1]{{\centering\hfill\break\hbox{\scriptsize Image Source:\thinspace{\small\itshape #1}}\par}}

\title{Using GitHub in First Year Computer Science Courses}


\author{Dr. Robert Lowe\\}

\institute[Maryville College] % (optional, but mostly needed)
{
  Division of Mathematics and Computer Science\\
  Maryville College
}

\date[]{}
\subject{}

\pgfdeclareimage[height=0.5cm]{university-logo}{images/Maryville-College}
\logo{\pgfuseimage{university-logo}}



\AtBeginSection[]
{
  \begin{frame}<beamer>{Outline}
    \tableofcontents[currentsection]
  \end{frame}
}


\begin{document}

\begin{frame}
  \titlepage
\end{frame}


% Structuring a talk is a difficult task and the following structure
% may not be suitable. Here are some rules that apply for this
% solution: 

% - Exactly two or three sections (other than the summary).
% - At *most* three subsections per section.
% - Talk about 30s to 2min per frame. So there should be between about
%   15 and 30 frames, all told.

% - A conference audience is likely to know very little of what you
%   are going to talk about. So *simplify*!
% - In a 20min talk, getting the main ideas across is hard
%   enough. Leave out details, even if it means being less precise than
%   you think necessary.
% - If you omit details that are vital to the proof/implementation,
%   just say so once. Everybody will be happy with that.

\begin{frame}{Motivation}
    \begin{itemize}
        \item I was already teaching revision control as a tool.
        \item More and more job applications included a place for
            your ``GitHub URL''.
        \item I also wanted to encourage students to play with open
            source code.
    \end{itemize}
\end{frame}

\begin{frame}{GitHub Academic Services}
    \begin{columns}
        \column{0.7\textwidth}
        \begin{itemize}
            \item \url{https://classroom.github.com}
            \item GitHub has a classroom interface.
            \item This creates an Organization, which houses all your
                repositories.
            \item Premium features, such as creating private
                repositories, are available for free to educators.
            \item The classroom interface creates a repository for
                each assignment.
        \end{itemize}
        \column{0.3\textwidth}
        \includegraphics[width=\textwidth]{classroom-logo}
    \end{columns}
\end{frame}

\begin{frame}{Initial Rollout in Introduction to CS II}
    \begin{itemize}
        \item Initially, I used the classroom interface.
        \item I gave 5 programming assignments, the first 4 in private
            repositories.
        \item The fifth assignment was a public repository, in this
            project the students were to write an original game.
        \item After CS II, their GitHub account would contain one open
            source program.
    \end{itemize}
\end{frame}

\begin{frame}{Management Scripts}
    \begin{itemize}
        \item I ultimately abandoned the classroom interface.
        \item I wrote a series of scripts to automate my workflow:
        \begin{itemize}
            \item deploy-assignment
            \item distribute
            \item distribute-grades
            \item fetch-assignment
            \item grade-assignment
            \item grade-report
        \end{itemize}
        \item These scripts are available at: 
        \newline\url{https://github.com/pngwen/cs-classroom}
    \end{itemize}
\end{frame}

\begin{frame}[fragile]{Using GitHub in Introduction to CS I}

    \begin{columns}
    \column{0.5\textwidth}
    \begin{itemize}
        \item Students have one private repository for the whole semester.
        \item Includes a \texttt{.gitignore} which will not publish
            binaries.
        \item Classroom examples are deployed in the example
            directory.
        \item Grade reports are published to their repositories.
        \item We do one group assignment at the end of the semester,
            where groups share a private GitHub repository.
    \end{itemize}

    \column{0.5\textwidth}
    \begin{block}{Student Repository}
    \begin{adjustbox}{max totalheight=0.8\textheight}
    \begin{BVerbatim}
GRADE.TXT
boilerplate.cpp
examples
    01-Intro-C++
    03-Prog-Structure
    04-Arithmetic
    05-Decisions
    06-Loops
    ...
programs
    program1
    program2
    ...
labs
    week2
    ...
    \end{BVerbatim}
    \end{adjustbox}
    \end{block}
    \end{columns}
\end{frame}


\begin{frame}{Goals in Both Semesters}
    \textbf{CSC111 - Intro CS I}
    \begin{itemize}
        \item Gain familiarity with the basic git workflow: pull, add,
            commit, push.
        \item Collaborate via GitHub.
        \item Easily access grades and examples.
    \end{itemize}

    \vspace{1cm}
    \textbf{CSC111 - Intro CS II}
    \begin{itemize}
        \item Learn to manage GitHub repositories.
        \item Treat each assignment as a new GitHub project.
        \item Publish an open source project via GitHub.
        \item Learn to play with code found on GitHub.
    \end{itemize}
\end{frame}

\begin{frame}{Results}
    \begin{itemize}
        \item Students seem to take to GitHub.
        \item I have seen better directory organization from my CS
            I students!
        \item In the first semester, they get an intuitive feel for
            collaboration.
        \item By the end of the second semester, they are able to
            participate in community projects.
    \end{itemize}
\end{frame}

\begin{frame}{Future Improvements}
\begin{itemize}
    \item I need to document my scripts.
    \item I need to make the scripts more user friendly.
    \item In the future, I may create a project that calls on students
        to pull together several open source projects and modify them.
    \item Slowly, but surely, I am migrating all my classroom sources
        to \url{https://github.com/relowe}
\end{itemize}
\end{frame}

\end{document}


